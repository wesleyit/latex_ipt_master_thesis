\section{INTRODUÇÃO}

Este é um exemplo de texto de introdução.

\subsection{Buscando Orientação}

No site do \textcite{texdoc_2023} é possível buscar por pacotes \LaTeX para obter informações de como utilizá-los.

\ABNTfigure
{Captura de tela do TEXDoc.}
{\textcite{texdoc_2023}}
{texdoc.png}
{width=0.8\textwidth}
{fig:texdoc}

É possível referenciar imagens usando labels, como a \ref{fig:texdoc}.

\subsubsection{Conformidade com ABNT}

A ABNT define que há diferenças entre quadros e tabelas.
informações quantitativas vão em tabelas, qualitativas vão em quadros.

\ABNTboard
{Deuses do Olimpo na Grécia e suas versões romanas.}
{Autor (2023).}
{board:deuses}
{
    \begin{tabular}{|l|l|l|l|}
        \hline
                        & \textbf{Amor}     & \textbf{Sabedoria} & \textbf{Guerra} \\ \hline
        \textbf{Grécia} & \textit{Afrodite} & \textit{Athena}    & \textit{Ares}   \\ \hline
        \textbf{Roma}   & \textit{Venus}    & \textit{Minerva}   & \textit{Marte}  \\ \hline
    \end{tabular}
}


\ABNTtable
{Crescimento populacional.}
{Autor (2023).}
{table:população}
{
    \begin{tabular}{llll}
        \hline
                        & \textbf{2000}        & \textbf{2010}        & \textbf{2020}        \\ \hline
        \textbf{Brasil} & \textit{200.000.000} & \textit{205.000.000} & \textit{212.000.000} \\ \hline
    \end{tabular}
}

As siglas e acrônimos podem adicionados ao arquivo acronyms.txt, como foi feito com \ac{abnt}.
