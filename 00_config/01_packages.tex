% One of the most important sections, the references:
\usepackage[english, brazilian]{babel}
\usepackage[backend=biber,backref=true,repeatfields=true,style=abnt]{biblatex}
\usepackage{csquotes}
\usepackage{quoting}
\usepackage{hyperref}

% This will help to customize captions:
\usepackage{caption}

% Use and create colors definitions (with tables support):
\usepackage[table,xcdraw]{xcolor}
% Now it is possible to define colors like ina a theme:
\definecolor{fgcolor}{HTML}{000000}
\definecolor{bgcolor}{HTML}{FFFFFF}
\definecolor{c1color}{HTML}{336699}
\definecolor{c2color}{HTML}{99CCFF}
\definecolor{c3color}{HTML}{CC9900}

% These packages help to improve the tables building:
\usepackage{array}  % Offers more flexible column formatting
\usepackage{booktabs}  % Supports professional looking tables
\usepackage{tabularx}  % Columns that expands to fill
\usepackage{multirow}  % Lets tabular material span multiple rows
\usepackage{array}  % Provides tabular that can split across pages
\usepackage{multicol}  % Creates text columns

% Tools for drawing boxes:
\usepackage{tcolorbox}

% This enables \fbox, \shadowbox, \doublebox, \ovalbox and \Ovalbox:
\usepackage{fancybox}

% Although this package requires xelatex or luatex,
% it allows to change fonts for the entire document
% using TTF files or fonts available in the SO:
\usepackage{fontspec}

% These fonts are used for special math symbols.
% American Mathematical Society packages:
\usepackage{amsmath}
\usepackage{amsfonts}
\usepackage{amsthm}
\usepackage{amssymb}

% These packages allow setting line spacing and margins:
\usepackage{geometry}  % This package allows selecting margins size
\usepackage{setspace}  % Allows selecting line spacing
\usepackage{indentfirst}  % Indents each first paragraph

% To have acronyms and glossaries working:
\usepackage[printonlyused]{acronym}

% In order to be compliant to ABNT, it is necessary to set some
% special values to titles, indexes, tables, captions, etc:
\usepackage{titlesec}  % Allows easily changing titles
\usepackage{tocloft}  % Allows creating and customizing tables of things

% The float package is necessary for creating custom float types,
% like boards, a new kind of table:
\usepackage{float}

% Insert pictures, make drawings and plots:
\usepackage{graphicx}
\graphicspath{{./03_images/}}
\usepackage{tikz}
\usepackage{pgfplots}
\pgfplotsset{compat=1.18}

% Better headers and footers:
\usepackage{fancyhdr}

% Better time and date printing:
\usepackage[portuges]{datetime2}


% =====================================================================
% Enable for debugging hbox errors:
% \usepackage{showframe}
