% =============================================================================
%% Packages -- Only necessary and tested packages here:


% =============================================================================
%% Language, links, lists and citations:
\usepackage[english, brazilian]{babel}  % pt_br and en_us support
\usepackage[backend=biber,backref=true,repeatfields=true,style=abnt]{biblatex}

%% These packages are required by biblatex, and improve its capabilities:
\usepackage{csquotes}
\usepackage{quoting}
\usepackage{hyperref}

% To have acronyms and glossaries working:
\usepackage[printonlyused]{acronym}


% =============================================================================
%% Fonts, headers, titles and captions:
\usepackage{caption}  % better captions on images, tables and boards

% These fonts are used for special math symbols.
% American Mathematical Society packages:
\usepackage{amsmath}
\usepackage{amsfonts}
\usepackage{amsthm}
\usepackage{amssymb}

% Although this package requires xelatex or luatex,
% it allows to change fonts for the entire document
% using TTF files or fonts available in the SO:
\usepackage{fontspec}

% In order to be compliant to ABNT, it is necessary to set some
% special values to titles, indexes, tables, captions, etc:
\usepackage{titlesec}  % Allows easily changing titles
\usepackage{tocloft}  % Allows creating and customizing tables of things

% Better time and date printing:
\usepackage[portuges]{datetime2}

% Better headers and footers:
\usepackage{fancyhdr}


% =============================================================================
%% Images, tables, boards, columns, margins, indent and related:
% Use and create colors definitions (with tables support):
\usepackage[table,xcdraw]{xcolor}  % allow creating custom colors

%% These packages help to improve the tables building:
\usepackage{array}  % Offers more flexible column formatting
\usepackage{booktabs}  % Supports professional looking tables
\usepackage{tabularx}  % Columns that expands to fill
\usepackage{multirow}  % Lets tabular material span multiple rows
\usepackage{array}  % Provides tabular that can split across pages
\usepackage{multicol}  % Creates text columns

%% Tools for drawing boxes:
\usepackage{tcolorbox}

%% This enables \fbox, \shadowbox, \doublebox, \ovalbox and \Ovalbox:
\usepackage{fancybox}

%% These packages allow setting line spacing and margins:
\usepackage{geometry}  % This package allows selecting margins size
\usepackage{setspace}  % Allows selecting line spacing
\usepackage{indentfirst}  % Indents each first paragraph

%% The float package is necessary for creating custom float types,
% like boards, a new kind of table:
\usepackage{float}

%% Insert pictures, make drawings and plots:
\usepackage{graphicx}
\graphicspath{{./03_images/}}
\usepackage{tikz}
\usepackage{pgfplots}
\pgfplotsset{compat=1.18}

%% Better formatting for source codes:
\usepackage{listings}


% =====================================================================
%% These are packages used for Development and Debugging:

%% Generate dummy text:
\usepackage{lipsum}

%% Show frames for debugging hbox errors:
% \usepackage{showframe}
