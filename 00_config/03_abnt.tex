% =============================================================================
%% ABNT Formatting


% =============================================================================
%% The allowed page size for the thesis is A4 with 3cm and 2cm margins:
\geometry{left=3cm, top=3cm, right=2cm, bottom=2cm}


% =============================================================================
%% All paragraphs should have 1.25cm indentation:
\setlength{\parindent}{1.25cm}


% =============================================================================
%% According to ABNT, the documents should be written in Arial or Times fonts
% only. The same font must follow the entire text.
\setmainfont{Times New Roman}  % Roman family, if necessary
\setsansfont{Arial}  % The sans serif general-purpose font
\setmonofont{Courier New}  % Monospaced font for code and commands
\urlstyle{same}  % Hyperlinks should honor the default font
\renewcommand{\familydefault}{\sfdefault}  % The default set to Sans Serif


% =============================================================================
%% Fonts sizes can differ according to sections. The allowed sizes are:
%
% - 14pt bold for titles
% - 12pt bold UPPERCASE centered for sections titles
% - 12pt bold TitleCase centered for subsections titles
% - 12pt bold TitleCase centered for subsubsections titles
% - 12pt regular for normal text
% - 10pt bold for captions titles
% - 10pt regular for captions
%
% Following these rules, only 3 sizes need to be used in this document.
% As the base size is 12pt, then:
%
% \footnotesize = 10pt
% \normalsize = 12pt
% \large = 14pt

\titleformat{\section}
{\newpage\centering\normalfont\normalsize\bfseries\uppercase}
{\thesection}
{1em}{}[]
\titlespacing{\section}{0pt}{2\bigskipamount}{\bigskipamount}

\titleformat{\subsection}
{\normalfont\normalsize\bfseries}
{\thesubsection}
{1em}{}[]
\titlespacing{\subsection}{0pt}{2\bigskipamount}{\bigskipamount}

\titleformat{\subsubsection}
{\normalfont\normalsize\bfseries}
{\thesubsubsection}
{1em}{}[]
\titlespacing{\subsubsection}{0pt}{2\bigskipamount}{\bigskipamount}


% =============================================================================
%% The titles for lists of tables, images, etc. must comply
% with the same formatting as sections:
\addto\captionsbrazilian{
    \renewcommand{\listfigurename}{\uppercase{Lista de Ilustrações}}
    \renewcommand{\listtablename}{\uppercase{Lista de Tabelas}}
    \renewcommand{\listboardname}{\uppercase{Lista de Quadros}}
    \renewcommand{\lstlistlistingname}{\uppercase{Lista de Códigos-fonte}}
    \renewcommand{\contentsname}{\section*{Sumário}}
}

%% The captions must contain a proper format. This helps those environments,
% such lstlistings, that doesn't have two captions:
% usage: \ABNTFont{Autor (2022)} | \ABNTFont{\textcite{GARCIA_1987}}
\newcommand{\ABNTFont}[1]{
    \begin{center}
        \footnotesize\textbf{Fonte:\space} #1.
    \end{center}
}

% =============================================================================
%% The summary must be left aligned without indent, and with dots
% connecting the section name and the page number. Three levels
% must be included: sections, subsections and subsubsections:
\setcounter{tocdepth}{3}
\cftsetindents{section}{0pt}{40pt}
\renewcommand{\cftsecleader}{\cftdotfill{\cftdotsep}}
\cftsetindents{subsection}{0pt}{40pt}
\renewcommand{\cftsubsecleader}{\cftdotfill{\cftdotsep}}
\cftsetindents{subsubsection}{0pt}{40pt}
\renewcommand{\cftsubsubsecleader}{\cftdotfill{\cftdotsep}}

%% This setting will change the section entries in the 
% summary in order to make them uppercase:
\makeatletter
\let\oldl@section\l@section
\renewcommand*\l@section[2]{\oldl@section{\uppercase{#1}}{#2}}
\makeatother


% =============================================================================
%% ABNT requires 10pt gray page numbers on top right:
\fancypagestyle{plain}{
    \fancyhf{}
    \fancyhead[R]{\color{gray}\footnotesize\thepage}
    \renewcommand{\headrulewidth}{0pt}
    \renewcommand{\footrulewidth}{0pt}
}


% =============================================================================
%% According to ABNT, there are a proper formatting for captions:
\DeclareCaptionLabelSeparator{abntdash}{\space--\space}
\captionsetup{labelsep=abntdash}
\captionsetup{font=footnotesize, labelfont=bf}
\captionsetup[figure]{name=Figura}
% \captionsetup[table]{name=Tabela}
% \captionsetup[board]{name=Quadrinho}


% =============================================================================
%% In ABNT, citations with more than 3 lines must be in smaller font (10pt)
% and with a 4cm block indent:
% ABNT citation for 3 or more lines:
\newenvironment{ABNTCiteLong}
{\setlength{\parindent}{0cm}
    \begin{quoting}[rightmargin=0cm,leftmargin=4cm]
        \begin{singlespace}
            \begin{footnotesize}
                }
                {
            \end{footnotesize}
        \end{singlespace}
    \end{quoting}
}


% =============================================================================
%% In Computer Science it is very common to present source code.
% This settings will format lstlistings properly:
\lstdefinestyle{ABNTCodeStyle}{
    backgroundcolor=\color{white},
    commentstyle=\itshape\color{darkgray},
    keywordstyle=\bfseries\color{magenta},
    numberstyle=\color{lightgray},
    stringstyle=\color{blue},
    basicstyle=\ttfamily\footnotesize\color{black},
    breakatwhitespace=false,
    breaklines=true,
    captionpos=t,
    frame=lines,
    identifierstyle=\bfseries\color{black},
    keepspaces=true,
    numbers=left,
    numbersep=20pt,
    showspaces=false,
    showstringspaces=false,
    showtabs=false,
    tabsize=2,
}
\lstset{style=ABNTCodeStyle}
\renewcommand{\lstlistingname}{Código-fonte}


% =============================================================================
%% In ABNT, tables are not closed, and they are used to express
% numerical data. To demonstrate qualitative information instead
% of quantitative, use a board. It is similar to a table, but has
% borders on all its sides. This environment defines a board
% structure, and also its counters, captions, and lists of boards:

\newcommand{\listboardname}{Lista de Quadros}
\newfloat{boards}{htb}{lob}
\floatname{boards}{Quadro}
\newenvironment{board}[1][]
{
    \begin{table}[!htb]
        \centering
        }
        {
    \end{table}
}

%% To make things easier, this shortcut can help to create boards with less effort:
% usage: \ABNTboard{caption}{reference}{label}{tabular}
\newcommand{\ABNTboard}[4]{
    \begin{board}[#1]
        \captionsetup{name=Quadro}
        \captionof{boards}{#1}
        \label{#3}
        #4
        \caption*{\textbf{Fonte:} #2}
    \end{board}
}

%% The same for tables:
% usage: \ABNTtable{caption}{reference}{label}{tabular}
\newcommand{\ABNTtable}[4]{
    \begin{table}
        \centering
        \caption{#1}
        \label{#3}
        #4
        \caption*{\textbf{Fonte:} #2}
    \end{table}
}

% =============================================================================
%% This will create links to references and URLs
% and also display in which page a reference is cited:
\DefineBibliographyStrings{brazilian}{
    backrefpage={Citado na página\space},
    backrefpages={Citado nas páginas\space}
}


% =============================================================================
%% This snippet make easy to create standardized images with captions:
% usage: \ABNTfigure{caption}{reference}{path}{size}{label}
\newcommand{\ABNTfigure}[5]{ % dasdsadasdsa
    \begin{figure}[!htb]
        \centering
        \caption{#1}
        \includegraphics[#4]{#3}
        \caption*{\textbf{Fonte:} #2}
        \label{#5}
    \end{figure}
}


% =============================================================================
%% This snippet helps to create message boxes for in-text notifications:
% usage: \MSGBOXProblem{title}{text}
\newcommand{\MSGBOXProblem}[2]{
    \begin{tcolorbox}[
            width=\textwidth,
            colupper={white},
            colback={red},
            title={\centering \textbf{#1}},
            colbacktitle={red},
            coltitle={white}
        ]
        #2
    \end{tcolorbox}
}


% =============================================================================
%% Package hyperref setup
% This will set metadata in the resulting PDF file.
% It also will properly set colors in the hyperlinks:
\makeatletter
\pdfstringdefDisableCommands{
\def\\{}
\def\texttt#1{<#1>}
}
\makeatother
\hypersetup{
    colorlinks=true,
    linkcolor=fgcolor,
    filecolor=fgcolor,
    citecolor=fgcolor,
    urlcolor=c1color,
    pdfauthor={\thsstudent},
    pdftitle={\thstitle},
    pdfsubject={\thstitle},
    pdfkeywords={\thskeywords},
    pdfpagemode=FullScreen
}


% =============================================================================
